\section{Results}
Key themes, subthemes, and exemplar quotations are
displayed in Table~\ref{table:theme}. Five key themes were found: (1)
providers, parents, and adolescents viewed the CDSSYD as helpful for treatment; (2) providers, parents, and
adolescents differed in their views of the trustworthiness
of the CDSS-YD, with adults having more trust in the
data-driven approach of the CDSS-YD and adolescents
having more trust in the provider’s expertise; (3) parents
and adolescents saw their providers as having a critical
role in the use the CDSS-YD; (4) providers, parents, and
adolescents expressed a desire to understand how the
questionnaire responses informed the CDSS-YD’s treatment recommendation; and (5) adolescents expressed
discomfort with sharing their questionnaire results with
the parents, and they expressed a desire for privacy when
reviewing the CDSS-YD results with the provider.

\begin{longtable}{p{6.5cm} p{8.5cm}}
  \caption{Key themes regarding provider, parent, and adolescent attitudes toward the CDSS-YD}
  \label{table:theme} \\
  \toprule
  \textbf{Themes} & \textbf{Subthemes} \\
  \midrule
  \endfirsthead

  \multicolumn{2}{c}%
  {{\bfseries Table \thetable\ (continued)}} \\
  \toprule
  \textbf{Themes} & \textbf{Subthemes} \\
  \midrule
  \endhead

  \bottomrule
  \endfoot

  Providers, parents, and adolescents viewed the CDSS-YD as helpful for treatment. & Providers, parents, and adolescents viewed the CDSS-YD as an informative and useful tool for personalized treatment planning. \\
  & Providers, parents, and adolescents believed that use of the CDSS-YD could foster a stronger therapeutic relationship. \\
  Providers, parents, and adolescents differed in their views of the trustworthiness of the CDSS-YD. & Providers and parents viewed the CDSS-YD as trustworthy because it is research-based and the treatment recommendation is data-driven. \\
  & Use of the CDSS-YD increased parents’ trust in the provider and their perception of provider expertise. \\
  & Youth viewed the CDSS-YD treatment recommendation as less trustworthy than a provider recommendation. \\
  Parents and adolescents saw their providers as having a critical role in the use of the CDSS-YD. & The provider’s opinion of the CDSS-YD treatment recommendation was a facilitator of families’ trust. \\
  & It is important for the CDSS-YD to be used in the context of a discussion with the provider. \\
  Providers, parents, and adolescents expressed a desire to understand how the questionnaire responses informed the CDSS-YD’s treatment recommendation. & \\
  Adolescents expressed discomfort with sharing their questionnaire results with the parents, and they expressed a desire for privacy when reviewing the CDSS-YD results with the provider. & \\
\end{longtable}
