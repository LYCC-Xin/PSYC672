\documentclass[a4paper,man,natbib,floatsintext,12pt]{apa7}

\usepackage[english]{babel} %character and hyphenation rules specific to the language you choose
%\usepackage[utf8x]{inputenc}
\usepackage{graphicx}
\usepackage{color}
\usepackage{tikz}
\usepackage{amsmath}
\usepackage{blindtext}
\usepackage{longtable}
\usepackage{booktabs}
\usepackage{tabularx} %great for APA-style Tables
\usepackage{siunitx} % Required for good table alignmen
\sisetup{
  round-mode          = places, % Rounds numbers
  round-precision     = 2, % to 2 places
}
\usepackage{multirow}
\usepackage{booktabs}
\usepackage{wrapfig}
\usepackage{booktabs} % For nice table rules
\usetikzlibrary{shapes,decorations,arrows,calc,arrows.meta,fit,positioning}
\tikzset{
    -Latex,auto,node distance =1 cm and 1 cm,semithick,
    latent/.style ={ellipse, draw, minimum width = 0.7 cm},
    observed/.style ={rectangle, draw},
    bidirected/.style={Latex-Latex,dashed},
    el/.style = {inner sep=2pt, align=left, sloped}
}
\newcommand{\sigFtest}[4]{\textit{F}(#1,#2) = #3, \textit{p}$<$#4}
\newcommand{\nonsigFtest}[3]{\textit{F}(#1,#2) = #3, \textit{p}$>$.05}


\title{Adolescent, parent, and provider attitudes toward a machine learning based clinical decision support system for selecting treatment for youth depression}
\shorttitle{CDSS-YD Attitudes}
\authorsnames[1,2,1,3,4,{3,5},5,5]{Meredith Gunlicks-Stoessel, Yangchenchen Liu, Catherine Parkhill, Nicole Morrell, Mimi Choy-Brown, Christopher Mehus, Joel Hetler, Gerald August}
\authorsaffiliations{{Department of Psychiatry \& Behavioral Sciences, University of Minnesota, 2025 E River Parkway, 55414 Minneapolis, MN, USA},{Department of Psychology, University of Minnesota, Minneapolis, MN, USA},{Center for Applied Research and Educational Improvement, University of Minnesota, St. Paul, MN, USA},{School of Social Work, University of Minnesota, St. Paul, MN, USA},{Department of Family Social Science, University of Minnesota, St. Paul, MN, USA}}
\journal{The Best Psychology Journal EVER!}
\abstract{Background: Machine learning based clinical decision support systems (CDSSs) have been proposed as a means
of advancing personalized treatment planning for disorders, such as depression, that have a multifaceted etiology,
course, and symptom profile. However, machine learning based models for treatment selection are rare in the field of
psychiatry. They have also not yet been translated for use in clinical practice. Understanding key stakeholder attitudes
toward machine learning based CDSSs is critical for developing plans for their implementation that promote uptake
by both providers and families.
Methods: In Study 1, a prototype machine learning based Clinical Decision Support System for Youth Depression
(CDSS-YD) was demonstrated to focus groups of adolescents with a diagnosis of depression (n=9), parents (n=11),
and behavioral health providers (n=8). Qualitative analysis was used to assess their attitudes towards the CDSS-YD.
In Study 2, behavioral health providers were trained in the use of the CDSS-YD and they utilized the CDSS-YD in a
clinical encounter with 6 adolescents and their parents as part of their treatment planning discussion. Following the
appointment, providers, parents, and adolescents completed a survey about their attitudes regarding the use of the
CDSS-YD. 
Results: All stakeholder groups viewed the CDSS-YD as an easy to understand and useful tool for making
personalized treatment decisions, and families and providers were able to successfully use the CDSS-YD in clinical
encounters. Parents and adolescents viewed their providers as having a critical role in the use the CDSS-YD, and
this had implications for the perceived trustworthiness of the CDSS-YD. Providers reported that clinic productivity
metrics would be the primary barrier to CDSS-YD implementation, with the creation of protected time for training,
preparation, and use as a key facilitator.
Conclusions: Machine learning based CDSSs, if proven effective, have the potential to be widely accepted tools for
personalized treatment planning. Successful implementation will require addressing the system-level barrier of having
sufficient time and energy to integrate it into practice.}
\keywords{Clinical decision support systems, Depression, Adolescents, Health care providers, Attitudes}
\authornote{We have no known conflict of interest to disclose. Correspondence concerns this article should be addressed to Meredith Gunlicks-Stoessel
mgunlick@umn.edu}
\leftheader{Alternate page header in man mode}

%-------------- END PREAMBLE  -------------------


\begin{document}

\maketitle  %Insert my APA style title page

\section{Introduction}

Depression among adolescents is becoming an increasingly critical public health concern. An estimated 4.1 million adolescents in the United States had at least one
major depressive episode in 2020 \citep{substance_abuse_and_mental_health_services_administration_key_2021}. This represents
17.0\% of the population of 12–17 year-olds and is an
increase from 11.3\% in 2014 and 8.7\% in 2005 \citep{substance_abuse_and_mental_health_services_administration_key_2021,mojtabai_national_2016}. The
rise in prevalence is compounded by the fact that while
treatment options exist, 30–50\% of youth who receive an
evidence-based treatment do not experience the intended
reduction in symptoms \citep{treatment_for_adolescents_with_depression_study_tads_team_fluoxetine_2004, mufson_randomized_2004}.


\section{Method}

\subsection{Participants}
Adolescents and parents/caregivers were recruited from
a clinical trial of treatments for depression in adolescents
conducted by the principal investigator. Families were
contacted if they had provided consent to be contacted
about future research opportunities. Nine adolescents
(age range=13–16, mean age=15.11, \textit{SD}=1.05) participated in the focus groups.
\subsection{Materials and Procedure}
A total of six focus groups were conducted: two with adolescents, two with parents, and two with providers. Semi-structured interview guides for each group were designed
by the research team for this study and were guided by
relevant constructs from the literature on CDSS implementation in other fields. The primary domains of
interest were perspectives on (1) the acceptability and
appropriateness of the CDSS-YD, (2) determinants of use
of the CDSS-YD, and (3) potential impact on treatment
processes.

\section{Results}
Key themes, subthemes, and exemplar quotations are
displayed in Table~\ref{table:theme}. Five key themes were found: (1)
providers, parents, and adolescents viewed the CDSSYD as helpful for treatment; (2) providers, parents, and
adolescents differed in their views of the trustworthiness
of the CDSS-YD, with adults having more trust in the
data-driven approach of the CDSS-YD and adolescents
having more trust in the provider’s expertise; (3) parents
and adolescents saw their providers as having a critical
role in the use the CDSS-YD; (4) providers, parents, and
adolescents expressed a desire to understand how the
questionnaire responses informed the CDSS-YD’s treatment recommendation; and (5) adolescents expressed
discomfort with sharing their questionnaire results with
the parents, and they expressed a desire for privacy when
reviewing the CDSS-YD results with the provider.

\begin{longtable}{p{6.5cm} p{8.5cm}}
  \caption{Key themes regarding provider, parent, and adolescent attitudes toward the CDSS-YD}
  \label{table:theme} \\
  \toprule
  \textbf{Themes} & \textbf{Subthemes} \\
  \midrule
  \endfirsthead

  \multicolumn{2}{c}%
  {{\bfseries Table \thetable\ (continued)}} \\
  \toprule
  \textbf{Themes} & \textbf{Subthemes} \\
  \midrule
  \endhead

  \bottomrule
  \endfoot

  Providers, parents, and adolescents viewed the CDSS-YD as helpful for treatment. & Providers, parents, and adolescents viewed the CDSS-YD as an informative and useful tool for personalized treatment planning. \\
  & Providers, parents, and adolescents believed that use of the CDSS-YD could foster a stronger therapeutic relationship. \\
  Providers, parents, and adolescents differed in their views of the trustworthiness of the CDSS-YD. & Providers and parents viewed the CDSS-YD as trustworthy because it is research-based and the treatment recommendation is data-driven. \\
  & Use of the CDSS-YD increased parents’ trust in the provider and their perception of provider expertise. \\
  & Youth viewed the CDSS-YD treatment recommendation as less trustworthy than a provider recommendation. \\
  Parents and adolescents saw their providers as having a critical role in the use of the CDSS-YD. & The provider’s opinion of the CDSS-YD treatment recommendation was a facilitator of families’ trust. \\
  & It is important for the CDSS-YD to be used in the context of a discussion with the provider. \\
  Providers, parents, and adolescents expressed a desire to understand how the questionnaire responses informed the CDSS-YD’s treatment recommendation. & \\
  Adolescents expressed discomfort with sharing their questionnaire results with the parents, and they expressed a desire for privacy when reviewing the CDSS-YD results with the provider. & \\
\end{longtable}


\section{Discussion}
The current feasibility studies collected multi-method
feedback from adolescents, parents, and behavioral
health providers on a computationally-based CDSS
that guides personalized treatment planning for youth
with depression. These studies provide support for the feasibility of the CDSS-YD, which is an important step
toward future effectiveness studies. Overall, all stakeholder groups liked the CDSS-YD. They found it easy to
understand and useful for making treatment decisions.
This was true for providers and families who viewed the
CDSS-YD during a focus group, as well as for those who
used it during a clinical encounter. They perceived the
CDSS-YD to provide clarity and direction for engaging
in treatment planning, which can otherwise often feel
like an ambiguous or “trial and error” process. Providers
reported liking that the CDSS-YD helped provide some
structure to the treatment planning process and made it
easy for them to explain the treatment recommendations
to the family. Parents and providers particularly liked
that the CDSS-YD was developed from research and
that the treatment recommendation was based on objective data, as opposed to an opinion, which could be perceived as biased. Parents also reported that providers’ use
of the CDSS-YD would increase their perception of the
providers’ expertise because it would indicate they were
knowledgeable about the most recent science. Of note,
some of the negative beliefs and attitudes towards CDSSs
that were identified in other studies were not identified
regarding the CDSS-YD, including the belief that the use
of CDSS would reduce providers’ professional autonomy
or interfere with the provider-patient therapeutic relationship. In fact, all stakeholder groups viewed the use of
the CDSS-YD as a way of strengthening the therapeutic
relationship.



\bibliography{reference_CDSS-YD.bib}

\end{document}
https://www.overleaf.com/project/68bc9b6676c27897cbfba1ae