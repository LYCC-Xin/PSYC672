\section{Discussion}
The current feasibility studies collected multi-method
feedback from adolescents, parents, and behavioral
health providers on a computationally-based CDSS
that guides personalized treatment planning for youth
with depression. These studies provide support for the feasibility of the CDSS-YD, which is an important step
toward future effectiveness studies. Overall, all stakeholder groups liked the CDSS-YD. They found it easy to
understand and useful for making treatment decisions.
This was true for providers and families who viewed the
CDSS-YD during a focus group, as well as for those who
used it during a clinical encounter. They perceived the
CDSS-YD to provide clarity and direction for engaging
in treatment planning, which can otherwise often feel
like an ambiguous or “trial and error” process. Providers
reported liking that the CDSS-YD helped provide some
structure to the treatment planning process and made it
easy for them to explain the treatment recommendations
to the family. Parents and providers particularly liked
that the CDSS-YD was developed from research and
that the treatment recommendation was based on objective data, as opposed to an opinion, which could be perceived as biased. Parents also reported that providers’ use
of the CDSS-YD would increase their perception of the
providers’ expertise because it would indicate they were
knowledgeable about the most recent science. Of note,
some of the negative beliefs and attitudes towards CDSSs
that were identified in other studies were not identified
regarding the CDSS-YD, including the belief that the use
of CDSS would reduce providers’ professional autonomy
or interfere with the provider-patient therapeutic relationship. In fact, all stakeholder groups viewed the use of
the CDSS-YD as a way of strengthening the therapeutic
relationship.
