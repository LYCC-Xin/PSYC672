\section{Introduction}

Depression among adolescents is becoming an increasingly critical public health concern. An estimated 4.1 million adolescents in the United States had at least one
major depressive episode in 2020 \citep{substance_abuse_and_mental_health_services_administration_key_2021}. This represents
17.0\% of the population of 12–17 year-olds and is an
increase from 11.3\% in 2014 and 8.7\% in 2005 \citep{substance_abuse_and_mental_health_services_administration_key_2021,mojtabai_national_2016}. The
rise in prevalence is compounded by the fact that while
treatment options exist, 30–50\% of youth who receive an
evidence-based treatment do not experience the intended
reduction in symptoms \citep{treatment_for_adolescents_with_depression_study_tads_team_fluoxetine_2004, mufson_randomized_2004}.
